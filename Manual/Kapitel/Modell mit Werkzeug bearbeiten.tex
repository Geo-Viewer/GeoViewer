\chapter{Modell mit Werkzeug bearbeiten}
Oben links befindet sich eine Werkzeigleiste, welche hilfreiche Funktionen bietet, zum interagieren mit einem 3D-Modell. Sie besteht aus einem Auswahl-, einem Messungs-, einem Bewegungs-, einem Skalierungs- und einem Rotations-Werkzeug.

\begin{figure}[th]
    \centering
    \includegraphics[width=\linewidth]{Kapitel/graphics/WerkzeugleisteNEU.png}
    \label{fig:enter-label}
\end{figure}
\section{Modell auswählen}
Das Auswahl-Werkzeug kann genutzt werden um 3D-Modelle oder andere Objekte auszuwählen. Ein ausgewähltes Modell wird mit einer farbigen Umrandung hervorgehoben.

1. Wählen Sie das Auswahlwerkzeug oben links aus.\\
\begin{figure}[th]
    \centering
    \includegraphics[width=\linewidth]{Kapitel/graphics/AuswahlwerkzeugNEU.png}
    \label{fig:enter-label}
\end{figure}2. Klicken Sie mit der linken Maustaste auf das Objekt, welches Sie auswählen wollen.

\newpage
\section{Modell bewegen}
Nachdem ein Modell ausgewählt wurde, kann das Bewegungs-Werkzeug genutzt werden um das Modell zu bewegen. Die Position kann hierbei in alle Richtungen verändert werden. Das Bewegungs-Werkzeug befindet sich oben links.

\begin{figure}[th!]
    \centering
    \includegraphics[width=\linewidth]{Kapitel/graphics/BewegungsWerkzeugNeu.png}
    \label{fig:enter-label}
\end{figure}

\subsection{Auf horizontaler Ebene bewegen}
1. Klicken Sie auf das Bewegungs-Werkzeug oben links.\\
2. Halten Sie die linke Maustaste gedrückt und bewegen Sie ihre Maus in die gewollte Richtung.

Das ausgewählte Modell bewegt sich automatisch entsprechend der Mausbewegung auf der horizontalen Ebene, bis Sie die linke Maustaste loslassen.

\subsection{Senkrecht zur horizontalen Ebene bewegen}
1. Klicken Sie auf das Bewegungs-Werkzeug oben links.\\
2. Halten Sie die linke Maustaste und die "ALT" Taste gedrückt. Bewegen Sie ihre Maus nach oben oder unten.

Das ausgewählte Modell bewegt sich automatisch entsprechend der Mausbewegung senkrecht zur horizontalen Ebene, bis Sie die linke Maustaste loslassen.


\section{Modell skalieren}
Nachdem ein Modell ausgewählt wurde, kann das Skalierungs-Werkzeug genutzt werden um Modelle oder andere Objekte zu skalieren. Es kann hierbei vergrößert oder verkleinert werden.

1. Wählen Sie das Skalierungs-Werkzeug oben links aus.

\begin{figure}[th]
    \centering
    \includegraphics[width=\linewidth]{Kapitel/graphics/SkalierungswerkzeugNEU.png}
    \label{fig:enter-label}
\end{figure}

2. Halten Sie die linke Maustaste gedrückt und bewegen Sie die Maus vom Modell weg oder zum Modell hin.

3. Lassen Sie die Maus bei gewünschter Position des Modells los.

Wenn Sie die Maus vom Zentrum des Modells wegbewegen, so wird das Modell größer.\\
Wenn Sie die Maus zum Zentrum des Modells hinbewegen, so wird das Modell kleiner.


\section{Modell rotieren}
Nachdem ein Modell ausgewählt wurde, kann das Rotations-Werkzeug genutzt werden um Modelle oder andere Objekte zu rotieren. Es kann hierbei je nach Mausbewegung mit oder gegen den Uhrzeigersinn rotieren.

1. Wählen Sie das Rotations-Werkzeug oben links aus.

\begin{figure}[th]
    \centering
    \includegraphics[width=\linewidth]{Kapitel/graphics/RotierungswerkzeugNEU.png}
    \label{fig:enter-label}
\end{figure}

2. Halten Sie die linke Maustaste gedrückt, während Sie ihre Maus nach links oder rechts bewegen.

3. Lassen Sie die Maus bei gewünschter Position des Modells los.

Das ausgewählte Modell bewegt sich automatisch entsprechend der Mausbewegung im oder gegen den Uhrzeigersinn.

\section{Feinjustierung}
Die Transformation (bewegen, skalieren und rotieren) des Modells kann auch langsamer erfolgen, um präziser zu arbeiten.\\
Das Vorgehen bleibt wie in den vorherigen Abschnitten bereits beschrieben. Sie müssen nur zusätzlich beim gedrückt halten der linken Maustaste die „Umschalt“-Taste gedrückt halten. Die Transformation entspricht dann der Hälfte der normalen Transformation.
