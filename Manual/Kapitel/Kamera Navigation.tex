\chapter{Kamera Navigation}


\section{Kamerabewegung mit Maus}
Um die Kamera zu bewegen, halten Sie die rechte Maustaste gedrückt und bewegen Sie gleichzeitig ihre Maus in die gewollte Richtung. Die Kamera bewegt sich automatisch zusammen mit dem Rotationszentrum in die entsprechende Richtung, bis Sie die rechte Maustaste wieder loslassen.


1. Halten Sie die rechte Maustaste gedrückt.\\
2. Bewegen Sie die Maus in die gewünschte Richtung.\\
3. Lassen Sie die rechte Maustaste wieder los.\\

\section{Kamerabewegung mit Tasten}
Um die Kamera mit Tasten zu bewegen, halten Sie die linke Maustaste gedrückt und drücken Sie gleichzeitig eine Bewegungstaste. Die Kamera bewegt sich automatisch zusammen mit dem Rotationszentrum in die entsprechende Richtung.

1. Halten Sie die linke Maustaste gedrückt.\\
2. Drücken Sie zusätzlich die passende Bewegungstaste (siehe Tabelle).

\begin{table}[H]
    \centering
    \begin{tblr}{
        colspec={c|l},
        hline{2}={solid}
    }
        Taste & Bewegung \\
        W & Vorwärts \\
        S & Rückwärts \\
        A & Links \\
        D & Rechts \\ 
        Q & Runter \\ 
        E & Hoch \\ 
        
    \end{tblr}
    
    \label{tab:tastenbelegung}
\end{table}
\newpage
\section{Kameradrehung}
Zum Drehen der Kamera, bewegen Sie ihre Maus in die gewollte Richtung während Sie die linke Maustaste gedrückt halten. Die Kamera dreht sich automatisch in die entsprechende Richtung um das Rotationszentrum, bis Sie die linke Maustaste wieder loslassen.


1. Halten Sie die linke Maustaste gedrückt.\\
2. Bewegen Sie die Maus in die gewünschte Richtung.\\
3. Lassen Sie die linke Maustaste wieder los.\\

\section{Kamerabewegung in Richtung des Rotationszentrums}
Um die Kamera zum Rotationszentrum hin oder vom Rotationszentrum weg zu bewegen, scrollen Sie mit dem Mausrad nach unten oder nach oben. Die Kamera bewegt sich entsprechend entlang der Gerade von
der Kamera zum Rotationszentrum.

Mausrad nach oben scrollen \textbf{=} zum Rotationszentrum hin bewegen\\
Mausrad nach unten scrollen \textbf{=} vom Rotationszentrum weg bewegen
