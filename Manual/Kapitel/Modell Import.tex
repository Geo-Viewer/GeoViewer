\chapter{Modell Import}


Sie befinden sich in der Hauptansicht und wollen ein 3D-Modell im OBJ-Format importieren.

1. Klicken Sie mit der Maus oben links auf "\textit{Object}".

\begin{figure}[th]
    \centering
    \includegraphics[width=\linewidth]{Kapitel/graphics/Modell Import 1,2.png}
    \label{fig:enter-label}
\end{figure}

\squeezeup
2. Klicken Sie auf "\textit{Import Objects}", was unter "\textit{Objects}" erscheint, nachdem Sie auf "\textit{Objects}" geklickt haben.\\

Nachdem Sie auf "\textit{Import Objects}" geklickt haben, öffnet sich ein Import Menü.\\

3. Klicken Sie auf "\textit{Select Object}" und wählen Sie eine OBJ-Datei aus.\\
4. Klicken Sie auf "\textit{Select Coordinates}" und wählen Sie eine Textdatei mit Koordinaten aus.\\
5. Klicken Sie anschließend auf "\textit{Confirm}" um das Modell zu laden.

\newpage
\begin{figure}[th]
    \centering
    \includegraphics[width=\linewidth]{Kapitel/graphics/Modell Import 4,5,6.png}
    \label{fig:enter-label}
\end{figure}

\textbf{Bemerkung:}\\
Es ist auch möglich nur ein Modell ohne Karte oder nur eine Karte ohne Modell zu laden. Wählen Sie dafür nur eine OBJ-Datei aus oder nur eine Textdatei mit Koordinaten aus und drücken Sie auf "\textit{Confirm}".
